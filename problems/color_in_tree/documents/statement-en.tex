% To add an image or include a .tex file you need to add
% \CWD
% to the relative (to the main document) path.
%
% Example:
% \begin{center}
%   \includegraphics{\CWD/images/example.pdf}
% \end{center}

You are given a tree with $n$ vertices numbered from $1$ to $n$ and rooted at vertex $1$. Initially, each vertex has a color $c_i$.\\

It is needed to perform $q$ queries of the following types:\\

\begin{enumerate}
    \item Update the colors of all vertices in the subtree of vertex $v$. For each vertex of the subtree replace its color by the formula: $c_i = (c_i + 1) \; mod \; 64$. Where $x \; mod \; y$ stands for the remainder that results of dividing $x$ by $y$.
    \item Count the number of vertices in the subtree of vertex $v$ with color $c$.
\end{enumerate}

%
% For input, use one of the following
%

\inputdesc{First line contains two integers $n$ and $q$ $(1 \leq n, q \leq 10^5)$, the number of vertices in the tree and the number of queries to perform.\\

    Second line contains $n$ integers $c_i$ $(0 \le c_i \le 63)$, the initial colors of the vertices.\\

    Next $N-1$ lines contain two integers $a$ and $b$ $(1\le a,b\le n)$ representing the edges of the tree.\\

    Last $q$ lines contain the description of queries in the format described below:\\
    \begin{itemize}
        \item[] $1$ $v$ : Update the colors of all vertices in the subtree of vertex $v$ $(1 \le v \le n)$.
        \item[] $2$ $v$ $c$ : Count the number of vertices in the subtree of vertex $v$ $(1 \le v \le n)$ with color $c$ $(0 \le c\le 63)$.
    \end{itemize}
}
%
% For output, use one of the following
%

\outputdesc{Print the solution for each query of type $2$. All solutions should be printed on a separate line following the same order given in the input.}
%\sampleio will look for files named sample-n.in and sample-n.sol (where n is 1, 2, 3...)
%in the documents directory and include them as samples.

\sampleio
