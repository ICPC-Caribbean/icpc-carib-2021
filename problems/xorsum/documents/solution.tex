% Para adicionar uma imagem ou incluir um arquivo .tex você precisa 
% adicionar \CWD
% caminho relativo (ao documento principal) do diretório.
% 
% Exemplo:
% \begin{figure}
%   \includegraphics{\CWD/imagens/exemplo.pdf}
% \end{figure}

Let's say we have values $a, b, c$ consecutive in the given array and we want to change $b$ to maximize the value of $(a \oplus b) + (b \oplus c)$ (note that $b$ wont be involved in any other term of the summation for the whole array). Since $\oplus$ is a bitwise operator we can proceed independently for each bit. Let's say we have for some bit $i$ the values $0$ and $0$ in $a$ and $c$ respectively, then it's more convenient to set the $i-$\textit{th} bit of $b$ to $1$. The same way we can check all four cases and construct the best value of $b$ for a given pair $a, c$.  

From the above statement we can get the best solution with only one changed element by just trying all possible triplets of consecutive elements.

Let's suppose now we want to change two elements, there are two cases:
\begin{enumerate}
	\item The two elements are consecutive: 
	
		Consider the values $a, b, c, d$ consecutive in our array, this operation is similar to the previous described, with the difference that now we want to get $b$ and $c$, given a pair $a, d$ in order to maximize $(a \oplus b) + (b \oplus c) + (c \oplus d)$. It can be solved again by checking all four cases on $a$'s and $d$'s bits, and then trying all possible $4-$tuples of consecutive elements.
	
	\item The two elements are not consecutive: (we can change its values independently of each other)
		
		Let's say we calculate $pref(i)$ as the best result we can get by changing only one element in the positions from $1$ to $i$ (eg. the $i-$\textit{th} prefix). We can easily do this by using some simple \textit{Dynamic Programming} technique. 
		
		Then suppose we want to change element $j$ as the second element; we just need to check the triplet $A[j-1], A[j], A[j+1]$ (where $A$ denotes the array) as previously described and consider the value of $pref(j-2)$ (since they must not be consecutive) as our base value for the summation. 
\end{enumerate}

Overall complexity is $\mathcal{O}(n \cdot \log M)$ where $M$ is the maximum value among the array's elements. The $\log$ factor is included because we check all bits for each element. 

\textbf{Note}: Special care need to be taken when changing elements in the first/last positions of the array.

