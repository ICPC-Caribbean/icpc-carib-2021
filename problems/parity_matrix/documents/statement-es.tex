% To add an image or include a .tex file you need to add
% \CWD
% to the relative (to the main document) path.
%
% Example:
% \begin{center}
%   \includegraphics{\CWD/images/example.pdf}
% \end{center}


Decimos que una matriz de números enteros es ``par'' si las sumas de los números por cada fila  y cada columna son números pares.\\

En los siguientes ejemplos, las matrices $A$, $B$ y $C$ son ``pares''. La matriz $D$ no es ``par'' debido a que su segunda columna y su última fila tienen sumas impares. La matriz $E$ no es ``par'' ya que sus dos columnas tienen sumas impares.\\

\begin{equation*}
  A =
  \begin{bmatrix}
    2
  \end{bmatrix},
  B =
  \begin{bmatrix}
    1 & 3 & 0 \\
    3 & 5 & 6
  \end{bmatrix},
  C =
  \begin{bmatrix}
    3 & 1 \\
    1 & 9
  \end{bmatrix},
  D =
  \begin{bmatrix}
    1 & 1 & 8 \\
    3 & 0 & 7 \\
    0 & 4 & 3
  \end{bmatrix},
  E =
  \begin{bmatrix}
    7 & 3
  \end{bmatrix}
\end{equation*}

Se tiene una matriz $A$, y se desea obtener una matriz ``par'' ejecutando la \textbf{mínima} cantidad de veces la siguiente operación:\\
\begin{itemize}
  \item Seleccionar una celda de la matriz e incrementar su valor en 1. Se permite seleccionar la misma celda en operaciones distintas.
\end{itemize}

Imprima cualquier matriz final ``par'' que resulte de ejecutar la operación anterior la menor cantidad de veces posible.

%
% For input, use one of the following
%

\inputdesc{La primera línea contiene dos números enteros $r$ y $c$ $(1 \le r, c \le 50)$. Las siguientes $r$ líneas definen la matriz $A$. Cada línea contiene $c$ números enteros con valores entre $0$ a $100$. El $j$\textit{-ésimo} número de la $i$\textit{-ésima} línea corresponde a la celda $A_{i, j}$ de la matriz.}

%
% For output, use one of the following
%

\outputdesc{Imprima cualquier matriz ``par'' resultante en el mismo formato indicado en la sección de entrada: $r$ filas con $c$ números en cada fila.}


%\sampleio will look for files named sample-n.in and sample-n.sol (where n is 1, 2, 3...)
%in the documents directory and include them as samples.

\sampleio
